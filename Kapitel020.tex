\chapter{Vorbetrachtungen}

\section{Anwendungsf"alle}

Um eine Anforderungsliste an die zu erstellende Software ausarbeiten zu k"onnen, soll zun"achst eine Liste von Anwendungsf"allen ausgearbeitet werden.
Aus der Liste h"aufiger Arbeitsabl"aufe und typischer Aufgabenstellungen wird ersichtlich, was der Nutzer von der Software erwartet.
Mithilfe dieser Erwartungen k"onnen im Anschluss die Anforderungen an das Programm formuliert und festgelegt werden.
% besonderheiten/merkmale der zu ueberpruefenden signale
% typische aufgabenstellungen
% haeufige arbeitsablaeufe

% enumerate auf a), b), usw aendern
\renewcommand{\theenumi}{\alph{enumi}}
\renewcommand{\labelenumi}{\theenumi )}
% und los geht's
Der Anwender m"ochte ...
\begin{enumerate}
	\item einen Datensatzes laden.
		  Dieser Datensatz umfasst mehrere (Bio-) Signale die sowohl mit einer konstanten Abtastrate erfasst wurden als auch Signale die nicht zu "aquidistanten Zeitpunkten abgetastet wurden.
	\item einen geladenen Datensatz mit allen "Anderungen speichern.
		  Hierbei sollen auch Einstellungen gespeichert werden, die die optische Pr"asentation wiederspiegeln.
	\item sich Informationen zu dem geladenen Datensatz und seinen beinhalteten Signalen anzeigen lassen und ver"andern.
	\item bestimmte Signale des Datensatzes ausw"ahlen und sich diese in ihrem Verlauf anzeigen lassen (Signalansicht).
		  Hierbei m"ochte er Bildschirmgr"o\ss e der einzelnen Ansichten ver"andern.
	\item die Signalansicht bez"uglich der Zeit- und der Amplitudenachse vergr"o\ss ern und verkleinern k"onnen (Zoomen).
		  Entlang der Zeitachse m"ochte er sie verschieben k"onnen (Scrollen).
		  Signaleverl"aufe die parallel aufgenommen wurden, sollen auch zusammen gescrollt werden.
	\item in einer Signalansicht mehrere Signale mit denselben Achsen darstellen lassen.
		  Beispielsweise um ein Roh- und ein verarbeitetes Signal miteinander vergleichen zu k"onnen.
	\item einen Amplitudenbereich eines Signals optisch hervorheben.
	\item einzelne Zeitpunkte im Signalverlauf mit einer Markierung versehen und kommentieren.
		  Diese Markierung kann sowohl f"ur ein bestimmtes Signal gelten, aber auch f"ur alle Signale des Datensatzes.
	\item einen Zeitabschnitt markieren. Die Markierung der Abschnitte soll analog zur Markierung von Zeitpunkten erfolgen.
	\item die Markierungen ver"andern (zeitlich verschieben, umbennen) oder l"oschen.
	\item Markierungen gemeinsam mit dem Datensatz aber auch unabh"angig vom Datensatz abspeichern.
\end{enumerate}

\section{Anforderungen}

Aus den oben genannten Anwendungsf"allen ergeben sich die folgenden Anforderungen an das Programm.
Das Programm ...
\renewcommand{\theenumi}{\Alph{enumi}}
\renewcommand{\labelenumi}{\theenumi )}
\newcommand{\AF}[1]{\item \label{AF:#1}}
\begin{enumerate}
	\AF{gui} muss eine grafische Benutzeroberfl"ache besitzen.
	\AF{datensatz} muss ein Datensatzformat unterst"utzen, das "aquidistant und nicht "aquidistante abgetastete Signale speichern kann.
	\AF{datenmanagement} soll in der Lage sein, Daten aus einem Datensatz zu laden.
						 Dem Nutzer muss es erm"oglicht werden, diese Signaldaten aus einer "Ubersicht auszuw"ahlen und in Diagrammen darstellen zu lassen.
						 Ferner muss der Benutzer eine M"oglichkeit haben, sich allgemeine Informationen des Datensatzes anzeigen zu lassen.
	\AF{diagramm} muss in der Lage sein, die Signalverl"aufe sowohl einzeln in einem Diagramm daszustellen, aber auch mehrere verschieden Signalverl"aufe in ein und demselben Diagramm zu visualisieren.
				  Diese Signalansichten sollten in ihrer Darstellungsgr"o\ss e durch den Nutzer ver"anderbar sein.
	\AF{ansicht} soll dem Benutzer erm"oglichen, seine Signalansicht frei "`bewegen"' zu k"onnen.
				 Es muss eine Vergr"o\ss erung und Verkleinerung bez"uglich der Abszissen- und der Ordinatenachse unterst"utzen.
				 Zus"atzlich ist die F"ahigkeit des Verschiebens der Ansicht gefordert.
				 Dabei sollen mehrere Diagramme gleichzeitig Verschoben werden k"onnen.
	\AF{amplitudenmarkierung} muss in der Lage sein einen Amplitudenbreich ein oder mehrerer Signalansichten optisch hervorzuheben.
	\AF{annotationen} soll dem Nutzer ein Werkzeug zur Verf"ugung stellen, das ihm erlaubt Datenpunkte zu annotieren.
					  Diese Annotationen sollen optisch in den Signalansichten ersichtlich sein und mit einem Kommentar versehen werden k"onnen.
					  Neben der Annotation einzelner Datenpunkte, soll auch die Markierung von Signalbereichen unterst"utzt werden.
					  Ferner ist gefordert, dass vorhandene Annotationen ver"anderbar sind.
	\AF{io} muss "Anderungen an den Signalen selbst und den Annotationen speichern k"onnen.
			Annotationen m"ussen unabh"angig von anderen Signalen gespeichert werden k"onnen.
			Insbesondere d"urfen Annotationen sich nicht ver"andern, wenn sich das Ursprungssignal ver"andert oder nicht mehr vorhanden ist.
	\AF{einstellungen} soll interne Einstellungen abspeichern und von einer Sitzung zur n"achsten "ubernehmen.
					   Optionen bez"uglich der Darstellung von Signalen sollen in dem Datensatz mit abgespeichert werden k"onnen.
\end{enumerate}

Die in der Aufgabenstellung geforderte Ausbauf"ahigkeit der Programms ist nicht durch die Anwendungsf"alle abgedeckt.
Daher wird die folgenden Anforderung nur auf Basis der Aufgabenstellung formuliert:
%Die folgende Anforderung l"asst sich nicht aus den Anwendungsf"allen erschlie\ss en, das die Erweiterbarkeit eine Forderung von zuk"unftigen Entwicklern ist und nicht die der Anwender.
%Um aber die Ausbauf"ahigkeit des Programms zu erm"oglichen, muss es 
\begin{enumerate}[resume]
	\AF{signalverarbeitung} Das Programm soll dem Benutzer erm"oglichen eine Signalverarbeitungsmethode auf ein gew"ahltes Biosignal anwenden zu k"onnen.
							Dabei muss das Originalsignal unver"andert bleiben.
							Der bearbeitete Signalverlauf kann als eigenes Signal im Datensatz abgespeichert werden.
							Die Implementierung dieser Anforderung soll beispielhaft f"ur zuk"unftige Entwickler erfolgen um die Erweiterbarkeit zu gew"ahrleisten.
\end{enumerate}

\section{Testszenarien}

% Szenarien aus der Anforderungsliste abgeleitet

%% EOF %%%%%%%%%%%%%%%%%%%%%%%%%%%%%%%%%%%%%%%%%%%%%%%%%%%%%%
