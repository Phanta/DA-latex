\chapter{Vorbetrachtungen}

\section{Anwendungsf"alle}

% besonderheiten/merkmale der zu ueberpruefenden signale
% typische aufgabenstellungen
% haeufige arbeitsablaeufe

% enumerate auf a), b), usw aendern
\renewcommand{\theenumi}{\alph{enumi}}
\renewcommand{\labelenumi}{\theenumi )}
% und los geht's
Der Anwender m"ochte ...
\begin{enumerate}
	\item einen Datensatzes laden. Dieser Datensatz umfasst mehrere (Bio-) Signale die sowohl mit einer konstanten Abtastrate erfasst wurden als auch Signale die nicht zu "aquidistanten Zeitpunkten abgetastet wurden.
	\item einen geladenen Datensatz mit allen "Anderungen speichern. Hierbei sollen auch Einstellungen gespeichert werden, die die optische Pr"asentation wiederspiegeln.
	\item sich Informationen zu dem geladenen Datensatz und seinen beinhalteten Signalen anzeigen lassen und ver"andern.
	\item bestimmte Signale des Datensatzes ausw"ahlen und sich diese in ihrem Verlauf anzeigen lassen (Signalansicht). Hierbei m"ochte er Bildschirmgr"o\ss e der einzelnen Ansichten ver"andern.
	\item die Signalansicht bez"uglich der Zeit- und der Amplitudenachse vergr"o\ss ern und verkleinern k"onnen (Zoomen). Entlang der Zeitachse m"ochte er sie verschieben k"onnen (Scrollen). Signaleverl"aufe die parallel aufgenommen wurden, sollen auch zusammen gescrollt werden.
	\item in einer Signalansicht mehrere Signale mit denselben Achsen darstellen lassen. Zum Beispiel um ein Roh- und ein verarbeitetes Signal miteinander vergleichen zu k"onnen.
	\item einen Amplitudenbereich eines Signals optisch hervorheben.
	\item einzelne Zeitpunkte im Signalverlauf mit einer Markierung versehen und kommentieren. Diese Markierung kann sowohl f"ur ein bestimmtes Signal gelten, aber auch f"ur alle Signale des Datensatzes.
	\item einen Zeitabschnitt markieren. Die Markierung der Abschnitte soll analog zur Markierung von Zeitpunkten erfolgen.
	\item die Markierungen ver"andern (zeitlich verschieben, umbennen) oder l"oschen.
	\item Markierungen gemeinsam mit dem Datensatz aber auch unabh"angig vom Datensatz abspeichern.
\end{enumerate}

\section{Anforderungen}

% konkrete liste der zu erfuellenden anforderungen

\section{Testszenarien}

% Szenarien aus der Anforderungsliste abgeleitet

%% EOF %%%%%%%%%%%%%%%%%%%%%%%%%%%%%%%%%%%%%%%%%%%%%%%%%%%%%%
