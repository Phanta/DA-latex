\chapter{Validierung}
\label{chap:validierung}

\section{Vorbereitung}

Der erste Schritt zur Validierung des umgesetzten Programms besteht darin, schon in \ml vorhandene Daten in das vom Programm genutzte Datenformat umzuwandeln.
Hierf"ur wird ein \mlNS-Skript erstellt, dass die Daten in das \usNS-Format wandelt.
Die Eingabeparameter des Skriptes sind der Name f"ur den Datensatz, das Verzeichnis in dem der Datensatz gespeichert werden soll und die Daten.
Die Daten werden als strukturierte Variable an das Skript "ubergeben, dessen Struktur der des zu erstellenden Datensatzes "ahnelt.
Bis auf der Datenart (\emph{Signal}, \emph{Value} oder \emph{Event}) werden durch das Skript alle notwendigen zus"atzlichen Parameter mit sinnvollen Werten vorbelegt.
Nur wenn die Parameter von diesen Standardwerten abweichen, m"ussen diese dem Skript "ubergeben werden.
Somit ist der Aufwand zum Exportieren von Daten in das \usNS-Format minimal.

%-- NOTIZ: Speicherplatzbedarf bei speicherung in \verb|int16| und \verb|double|
%-- Skripte zur Wandlung Matlab <--> Unisens

\section{Erf"ullung der Anforderungen}
\label{sec:anforderungsvalidierung}

Zur "Uberpr"ufung des Programms

-- BONUS:
- Schreibschutz f"ur Annotationen
- Unterst"utzung von "`Echtzeit"'-Signalverarbeitung

\section{Evaluation der Nutzeroberfl"ache}



%\section{Validierung anhand der Annotation von fetalen \ac{EKG}-Daten}

%\section{Validierung mittels der Annotation"uberpr"ufung von ...}


%% EOF %%%%%%%%%%%%%%%%%%%%%%%%%%%%%%%%%%%%%%%%%%%%%%%%%%%%%%
