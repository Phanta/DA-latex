\chapter{Validierung}
\label{chap:validierung}

\section{Vorbereitung}

Der erste Schritt zur Validierung des umgesetzten Programms besteht darin, schon in \ml vorhandene Daten in das vom Programm genutzte Datenformat umzuwandeln.
Hierf"ur wird ein \mlNS-Skript erstellt, dass die Daten in das \usNS-Format wandelt.
Die Eingabeparameter des Skriptes sind der Name f"ur den Datensatz, das Verzeichnis in dem der Datensatz gespeichert werden soll und die Daten.
Die Daten werden als strukturierte Variable an das Skript "ubergeben, dessen Struktur der des zu erstellenden Datensatzes "ahnelt.
Bis auf der Datenart (\emph{Signal}, \emph{Value} oder \emph{Event}) werden durch das Skript alle notwendigen zus"atzlichen Parameter mit sinnvollen Werten vorbelegt.
Nur wenn die Parameter von diesen Standardwerten abweichen, m"ussen diese dem Skript "ubergeben werden.
Somit ist der Aufwand zum Exportieren von Daten in das \usNS-Format minimal.

%-- NOTIZ: Speicherplatzbedarf bei speicherung in \verb|int16| und \verb|double|
%-- Skripte zur Wandlung Matlab <--> Unisens

\section{Erf"ullung der Anforderungen}
\label{sec:anforderungsvalidierung}

Im \secref{testszenarien} spezifizierten Testszenarien werden genutzt um das erstellte Programm dahingehend zu pr"ufen, ob es die in \secref{anforderungen} genannten Anforderungen erf"ullen kann.
Da es sich bei den Testszenarien 1 bis 5 um einfach "`Schritt-f"ur-Schritt"'-Anleitungen handelt, wird in diesem Abschnitt auf eine genaue Beschreibung der Durchf"uhrung verzichtet.
Die aufgestellten Testszenarien werden nach den beschriebenen Anleitungen durchgef"uhrt.

\begin{table}[htb]
\caption{Ergebnis der durchgef"uhrten Testszenarien}
\label{tab:test_ergebnisse}
\centering
\begin{tabular}{|r|c|c|c|c|c|c|c|c|c|c|c|c|c|c|c|c|c|}
	\cline{2-14}
	\multicolumn{1}{r|}{} & \multicolumn{13}{|c|}{Anforderung} \\ \cline{2-14}
	\multicolumn{1}{r|}{} & A & B1 & B2 & C1 & C2 & D1 & D2 & E1 & E2 & E3 & F1 & F2 & F3 \\ \hline
	erf"ullt & \ding{51} & \ding{51} & \ding{51} & \ding{51} & \ding{51} & \ding{51} & \ding{51} & \ding{51} & \ding{51} & \ding{51} &  & \ding{51} & \ding{51} \\ \hline
	nicht erf"ullt & & & & & & & & & & & (\ding{55}) & &  \\ \hline %\XSolid
\end{tabular}

\vspace{2ex}

\begin{tabular}{|r|c|c|c|c|c|c|c|c|c|c|c|c|c|}
	\cline{2-14}
	\multicolumn{1}{r|}{} & \multicolumn{13}{|c|}{Anforderung} \\ \cline{2-14}
	\multicolumn{1}{r|}{} & G1 & G2 & G3 & G4 & H1 & H2 & H3 & H4 & I1 & I2 & J1 & J2 & J3 \\ \hline
	erf"ullt & \ding{51} & \ding{51} & \ding{51} & \ding{51} & \ding{51} & \ding{51} & \ding{51} & \ding{51} & & & \ding{51} & \ding{51} & \ding{51} \\ \hline
	nicht erf"ullt & & & & & & & & & \ding{55} & \ding{55} & & & \\ \hline
\end{tabular}
\end{table}

Vielmehr sollen hier die Ergebnisse der Testdurchf"uhrung pr"asentiert werden.
In \tabref{test_ergebnisse} sind diese "uberschtlich dargestellt.
Es ist ersichtlich, dass drei der 26 (Teil-) Anforderungen nicht erf"ullt wurden.
Im Folgendem sollen die Umst"ande er"ortert werden, warum diese Anforderungen nicht erf"ullt wurden.

\begin{description}
	\item[Anforderung F1] "`Das Programm soll dem Benutzer erm"oglichen, seine Signalansicht frei "`bewegen"' zu k"onnen.
				 Es muss eine Vergr"o\ss erung und Verkleinerung bez"uglich der Abszissen- und der Ordinatenachse unterst"utzen. [...]"'
\end{description}

- nicht unbedingt notwendig da Autorangeing der Achse
- "Uberladung der Nutzereingabe (steigende Komplexit"at)

\begin{description}
	\item[Anforderung K] "`Das Programm soll interne Einstellungen abspeichern und von einer Sitzung zur n"achsten "ubernehmen (1).
					   Optionen bez"uglich der Darstellung von Signalen sollen in dem Datensatz mit abgespeichert werden k"onnen (2)."'
\end{description}

- hoher programmiertechnischer Aufwand
- Abspeichern der visuellen Einstellungen innerhalb des Datensatzes nicht gew"unscht (was haben die im Datensatz verloren?)
- extern Abspeichern auch nicht sinnvoll, da Daten in meisten F"allen nie wieder ben"otigt werden

-- BONUS:
- Schreibschutz f"ur Annotationen
- Unterst"utzung von "`Echtzeit"'-Signalverarbeitung

\section{Evaluation der Nutzeroberfl"ache}



%\section{Validierung anhand der Annotation von fetalen \ac{EKG}-Daten}

%\section{Validierung mittels der Annotation"uberpr"ufung von ...}


%% EOF %%%%%%%%%%%%%%%%%%%%%%%%%%%%%%%%%%%%%%%%%%%%%%%%%%%%%%
