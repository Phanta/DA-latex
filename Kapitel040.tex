\chapter{Validierung}
\label{chap:validierung}

\section{Vorbereitung}

Der erste Schritt zur Validierung des umgesetzten Programms besteht darin, schon in \ml vorhandene Daten in das vom Programm genutzte Datenformat umzuwandeln.
Hierf"ur wird ein \mlNS-Skript erstellt, dass die Daten in das \usNS-Format wandelt.
Die Eingabeparameter des Skriptes sind der Name f"ur den Datensatz, das Verzeichnis in dem der Datensatz gespeichert werden soll und die Daten.
Die Daten werden als strukturierte Variable an das Skript "ubergeben, dessen Struktur der des zu erstellenden Datensatzes "ahnelt.
Bis auf der Datenart (\emph{Signal}, \emph{Value} oder \emph{Event}) werden durch das Skript alle notwendigen zus"atzlichen Parameter mit sinnvollen Werten vorbelegt.
Nur wenn die Parameter von diesen Standardwerten abweichen, m"ussen diese dem Skript "ubergeben werden.
Somit ist der Aufwand zum Exportieren von Daten in das \usNS-Format minimal.

%-- NOTIZ: Speicherplatzbedarf bei speicherung in \verb|int16| und \verb|double|
%-- Skripte zur Wandlung Matlab <--> Unisens

\section{Erf"ullung der Anforderungen}
\label{sec:anforderungsvalidierung}

Im \secref{testszenarien} spezifizierten Testszenarien werden genutzt um das erstellte Programm dahingehend zu pr"ufen, ob es die in \secref{anforderungen} genannten Anforderungen erf"ullen kann.
Da es sich bei den Testszenarien 1 bis 5 um einfach "`Schritt-f"ur-Schritt"'-Anleitungen handelt, wird in diesem Abschnitt auf eine genaue Beschreibung der Durchf"uhrung verzichtet.
Die aufgestellten Testszenarien werden nach den beschriebenen Anleitungen durchgef"uhrt.

\begin{table}[htb]
\caption{Ergebnis der durchgef"uhrten Testszenarien}
\label{tab:test_ergebnisse}
\centering
\begin{tabular}{|r|c|c|c|c|c|c|c|c|c|c|c|c|c|c|c|c|c|}
	\cline{2-14}
	\multicolumn{1}{r|}{} & \multicolumn{13}{|c|}{Anforderung} \\ \cline{2-14}
	\multicolumn{1}{r|}{} & A & B1 & B2 & C1 & C2 & D1 & D2 & E1 & E2 & E3 & F1 & F2 & F3 \\ \hline
	erf"ullt & \ding{51} & \ding{51} & \ding{51} & \ding{51} & \ding{51} & \ding{51} & \ding{51} & \ding{51} & \ding{51} & \ding{51} &  & \ding{51} & \ding{51} \\ \hline
	nicht erf"ullt & & & & & & & & & & & (\ding{55}) & &  \\ \hline %\XSolid
\end{tabular}

\vspace{2ex}

\begin{tabular}{|r|c|c|c|c|c|c|c|c|c|c|c|c|c|}
	\cline{2-14}
	\multicolumn{1}{r|}{} & \multicolumn{13}{|c|}{Anforderung} \\ \cline{2-14}
	\multicolumn{1}{r|}{} & G1 & G2 & G3 & G4 & H1 & H2 & H3 & H4 & I1 & I2 & J1 & J2 & J3 \\ \hline
	erf"ullt & \ding{51} & \ding{51} & \ding{51} & \ding{51} & \ding{51} & \ding{51} & \ding{51} & \ding{51} & & & \ding{51} & \ding{51} & \ding{51} \\ \hline
	nicht erf"ullt & & & & & & & & & \ding{55} & \ding{55} & & & \\ \hline
\end{tabular}
\end{table}

Vielmehr sollen hier die Ergebnisse der Testdurchf"uhrung pr"asentiert werden.
In \tabref{test_ergebnisse} sind diese "uberschtlich dargestellt.
Es ist ersichtlich, dass drei der 26 (Teil-) Anforderungen nicht erf"ullt wurden.
Im Folgendem sollen die Umst"ande er"ortert werden, warum diese Anforderungen nicht erf"ullt wurden.

\begin{description}
	\item[Anforderung F1] "`Das Programm soll dem Benutzer erm"oglichen, seine Signalansicht frei "`bewegen"' zu k"onnen.
				 Es muss eine Vergr"o\ss erung und Verkleinerung bez"uglich der Abszissen- und der Ordinatenachse unterst"utzen. [...]"'
\end{description}

Anforderung F1 wurde nur teilweise nicht erf"ullt.
Der Anwender kann die Ansicht bez"uglich der Abszisse frei verschieben.
Diese M"oglichkeit hat er nicht bez"uglich der Ordinate.
Die Skala der Ordinate wird automatisch an den Wertebereich der dargestellten Signale angepasst.
Diese automatische Anpassung ist f"ur die Darstellung der Signale und das Annotieren dieser ausreichend und komfortabel.
Daher ist diese Option, die zus"atzlich die Bedienung komplexer werden lie{\ss}e, nicht implementiert.

%- nicht unbedingt notwendig da Autorangeing der Achse
%- "Uberladung der Nutzereingabe (steigende Komplexit"at)

\begin{description}
	\item[Anforderung K] "`Das Programm soll interne Einstellungen abspeichern und von einer Sitzung zur n"achsten "ubernehmen (1).
					   Optionen bez"uglich der Darstellung von Signalen sollen in dem Datensatz mit abgespeichert werden k"onnen (2)."'
\end{description}

Die Anforderung K wurde nicht umgesetzt, weil der Komfortgewinn f"ur den Nutzer den erh"ohten Implementierungsaufwand nicht rechtfertigen.
Zudem erscheint in einer nachtr"aglichen Betrachtung die Abspeicherung von Anzeigeparametern \emph{innerhalb} des Datensatzes nicht sinnvoll.
Der Mehrwert einer solchen Option ist f"ur den Nutzer vorhanden, aber diese programmspezifischen Einstellungen sollten nicht Teil des Datensatzes sein.
Eine Speicherung der Daten au{\ss}erhalb des Datensatzes ist m"oglich, f"uhrt aber zu weiteren Problemen.
Wenn Anzeigeeinstellungen unabh"angig der Datens"atzen gespeichert werden, werden Informationen f"ur Datens"tze aufgehoben, die entweder nicht mehr genutzt werden oder nicht mehr vorhanden sind.
Zudem muss eine Identifizierung von Datens"atzen realisiert werden, die umbenannte oder auf dem Speichermedium verschobene Datens"atze handhaben kann.

%- hoher programmiertechnischer Aufwand
%- Abspeichern der visuellen Einstellungen innerhalb des Datensatzes nicht gew"unscht (was haben die im Datensatz verloren?)
%- extern Abspeichern auch nicht sinnvoll, da Daten in meisten F"allen nie wieder ben"otigt werden

W"ahrend des Testens der Applikation wurde festgestellt das es m"oglich sein soll "Anderungen bestimmter Annotationskan"ale durch den Nutzer zu verhindern.
Dies ist z.~B. der Fall, wenn in einem Datensatz ein Kanal mit Referenzannotierungen vorhanden ist.
Dazu kann die Datendatei innerhalb des Verzeichnisses als schreibgesch"utzt markiert.
Vom Programm wird dieser Zustand erkannt und der Nutzer werden diese Informationen auch innerhalb der \ac{GUI} dargestellt.
Zus"atzlich dazu werden die betroffenen Eintr"age dem Nutzer auch nicht mehr bei der Auswahl des zu editierenden Annotationskanals angezeigt.
Eine Fehlbedienung durch den nutzer wird somit verhindert.
Eine weitere Programmfunktionalit"at die nicht gefordert ist, ist die Signalverarbeitung in "`Echtzeit"'.
(Echtzeit im Sinne von schnell genug f"ur den Nutzer, fernab harter Echtzeit.)
Durch die F"ahigkeit unmittelbar "Anderungen in abgeleiteten Signalen zu pr"asentieren, wird der Nutzen des Programms f"ur den Nutzer deutlich erh"oht.

%-- BONUS:
%- Schreibschutz f"ur Annotationen
%- Unterst"utzung von "`Echtzeit"'-Signalverarbeitung

\section{[WIP]Anwendbarkeit des Programms in unterschiedlichen Problembereichen}

- einheitliches Format als einzige Bedingung
- sofern Daten in das Format passen, kann das programm genutzt werden
- Konvertierung der Daten "uber die bereit gestellten \mlNS-Skripte
- wird bereits an mehreren Stellen genutzt

%\section{Validierung anhand der Annotation von fetalen \ac{EKG}-Daten}

%\section{Validierung mittels der Annotation"uberpr"ufung von ...}


%% EOF %%%%%%%%%%%%%%%%%%%%%%%%%%%%%%%%%%%%%%%%%%%%%%%%%%%%%%
