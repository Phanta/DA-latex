\chapter{Auswertung und Diskussion}

\section{Bewertung der Evaluation}

Die Evaluation zeigt, dass das in dieser Arbeit erstellte Programm zur Analyse und Aufbereitung biomedizinischer Messdaten genutzt werden kann.
Durch das gew"ahlte Datensatzformat \us k"onnen Messdaten aus unterschiedlichen Aufgabenbereichen bearbeitet werden.
Dank der universellen Natur von \us wird eine problemunabh"angige Datenverarbeitung erm"oglicht.
Die nicht erf"ullten Anforderungen "`F1"' und "`K"' sind in der Praxis Anforderungen an den Nutzerkomfort.
Die Kernfunktionalit"at der Software ist durch sie nicht beeintr"achtigt.

\section{Ausblick}

Bei der Implementierung des Programms wurde auf ein einheitliches Dialogdesign geachtet.
Dennoch weichen einige Dialoge vom homogenen Layout ab.
In zuk"unftigen Versionen kann dieses Problem behoben werden, in dem f"ur die Dialoge eine gemeinsame Superklasse erstellt wird.
Diese soll Aufgaben der Darstellung und Standardschaltfl"achen (wie "`OK"' und "`Abbrechen"') bereitstellen, damit ein wirkliches Standardlayout geschaffen wird.
Zus"atzlich kann damit der Implementierungsaufwand, der f"ur grafische Komponenten besteht, reduziert werden.

\begin{figure}[bth]
\centering
\includegraphics[angle=-90, width=\textwidth]{bilder/detail_LiveSignalProcessor_Menus_interaction.pdf}
\caption{Interaktion zwischen \class{LiveSignalProcessor} und \class{Menus}}
\label{pic:detail_signalprocessing}
\end{figure}

Ein weiterer Ansatzpunkt f"ur Verbesserungen an dem bestehendem Programm ist die Implementierung der \code{signalprocessing}-\code{Menus}-Interaktion.
In der vorliegenden Version ist es notwendig, die einzelnen Konfigurationsdialoge der Signalverarbeitungsroutinen im Men"u zu verankern (vgl. \picref{detail_signalprocessing}).
Der entsprechende Men"uunterpunkt und der Quelltext zum Aufrufen des Konfigurationsdialoges muss durch den Algorithmusimplementierer erfolgen.
Die Signalverarbeitung selbst wird durch den Dialog gestartet.
Die Beendigung dieser erfolgt durch den Benutzer "uber das Men"u.
Eine Alternative dazu ist in \picref{better_detail_signalprocessing} dargestellt.
\class{ImplementationClass} ist die den Konfigurationsdialog implementierende Klasse.
Daf"ur muss ein \emph{Interface} \class{LiveSignalProcessorDialog} und entsprechender Code im \code{ui} umgesetzt werden.
Die Idee besteht darin, dass in der Klasse \class{Menus} die Methode \code{registerLSPDialog()} automatisch einen Men"ueintrag f"ur einen \class{LiveSignalProcessor} erzeugt.
Wird dieser Eintrag durch den Nutzer gew"ahlt, wird der Konfigurationsdialog mit der \emph{Interface}-Methode \code{show()} ge"offnet.
Als R"uckgabe wird ein \class{LiveSignalProcessor} erwartet.
Dieser kann anschlie{\ss}end durch \class{Menus} gestartet und gestoppt werden.
Somit w"urde der gesamte Code zur Integration des Konfigurationsdialoges und der Signalverarbeitungsmethode in das Programm auf eine Zeile reduziert werden:
\code{Menus.getInstance().registerLSPDialog(new ImplementationClass())}.
Damit wird der Implementierungsaufwand f"ur neue Signalverarbeitungsmethoden auf den Konfigurationsdialog und den Algorithmus selbst reduziert.

\begin{figure}[htb]
\centering
\includegraphics[angle=-90, width=\textwidth]{bilder/verbesserung_LiveSignalProcessor_Menus_interaction.pdf}
\caption{Verbesserte Interaktion zwischen \class{LiveSignalProcessor} und \class{Menus}}
\label{pic:better_detail_signalprocessing}
\end{figure}

%- andere \code{signalprocessing} Umsetzung
%- freie H"ohen"anderung der Signalansichten durch den Nutzer
%% EOF %%%%%%%%%%%%%%%%%%%%%%%%%%%%%%%%%%%%%%%%%%%%%%%
