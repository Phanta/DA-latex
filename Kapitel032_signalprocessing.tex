\section{Signalverarbeitung im Paket \texttt{signalprocessing}}
\label{sec:signalprocessing}

\subsection{"Uberblick "uber das \code{signalprocessing}-Paket}

Die Signalverarbeitung der Software wird durch das Paket \class{signalprocessing} erm"oglicht.
Der Schwerpunkt bei der Implementierung dieses Paketes liegt in der Umsetzung einer erweiterbaren Plattform f"ur Signalverarbeitungsfunktionen.
Die Bereitstellung und Implementierung konkreter Signalverarbeitungsfunktion spielen eine untergeordnete Rolle.
Die Verwendung des Paketes soll an dieser Stelle nur beispielhaft erfolgen.
Als Musterbeispiel f"ur die Funktionalit"at der Signalverarbeitung wird die Berechnung von RR-Zeitreihen herangezogen.
Im Detail soll aus den Zeitpunkten von Annotationen eines Kanals eine Zeitreihe der zeitlichen Abst"ande der gegebenen Annotationen automatisiert erstellt werden.
Es wird bewusst eine Signalverarbeitungsfunktion geringer Komplexit"at implementiert um den Fokus auf das allgemeine Zusammenspiel der beteiligten Klassen und nicht auf den Algorithmus selbst zu legen.

Zur Berechnung der Schlag-zu-Schlag-Intervalle werden folgende Schritte ausgef"uhrt:
\begin{enumerate}
	{% Klammern damit "Anderungen lokal bleiben
	\renewcommand{\theenumi}{\arabic{enumi}}
	\renewcommand{\labelenumi}{{\theenumi}.}
	\item Auswahl eines Annotationskanals durch den Benutzer
	\item Berechnung der Zeitpunkte der Annotationen aus ihren Samplenummern und der (virtuellen) Abtastfrequenz
	\item Berechnung der Differenz der Zeitpunkte aufeinander folgender Annotationen
	\item Speichern der errechneten Zeitdifferenzen in einem durch den Benutzer benannten Signal
	}
\end{enumerate}
Um die Darstellung der Zeitreihe zu erm"oglichen, wird jeder errechneten Differenz ein Zeitpunkt zugewiesen.
Die errechneten Datenpunkte werden als \class{Value} gespeichert.
Der Wert $V_{value}$ ist die Differenz der Zeitpunkte $t_{n+1} - t_n$ zweier Annotationen $A_n$ und $A_{n+1}$.
Der Zeitpunkt des \class{Value}s $t_{value}$ wird auf den Zeitpunkt $t_{n+1}$ der zweiten Annotation $A_{n+1}$ festgelegt.
Die Implementierung dieses Algorithmus ist im Paket \code{rrcalc} mit den zwei Klassen \class{RRCalculator} und \class{LiveRRCalculator} realisiert.

\begin{figure}[htb]
\centering
\includegraphics[angle=-90, width=\textwidth]{bilder/package_signalprocessing.pdf}
\caption{"Ubersicht "uber das \code{signalprocessing}-Paket}
\label{pic:package_signalprocessing}
\end{figure}

In dem \code{signalprocessing}-Paket wird zwischen zwei Arten der Ausf"uhrung von Signalverarbeitungsfunktionen unterschieden (siehe \picref{package_signalprocessing}).
Die erste Art ist das einmalige Anwenden einer Verarbeitungsfunktion auf ein oder mehrerer Signale und ist im \secref{signal_processor} er"ortert.
Ein Anwendungsbeispiel daf"ur sind Filterfunktionen oder die Berechnung von RR-Intervallen aus bereits annotierten \ac{EKG}-Daten.
Die zweite Ausf"uhrungsart ist eine kontinuierliche Verarbeitung von Signalen.
Hierbei wird auf eine Ver"anderung des zu verarbeitenden Signals reagiert und die Ausgabe entsprechend angepasst.
Das implementierte Beispiel daf"ur ist die Berechnung und Anzeige der RR-Zeitreihe, w�hrend der Benutzer des Programms Annotationen ver"andert bzw. korrigiert.
Diese zweite Verarbeitungsart ist im \secref{live_signal_processor} weiter unten beschrieben.

%- zwei Arten von Signalverarbeitern unterst"utzt: einmaliger und kontinuierlicher Ablauf

\subsection{Einmalige Signalverarbeitung durch \class{SignalProcessor}}
\label{sec:signal_processor}

Die Klasse \class{SignalProcessor} stellt einen Prototypen einer allgemeinen Signalverarbeitungsfunktion dar.
Objekte dieser Klasse besitzen ein oder mehrere Eingangssignale und ebenso ein oder mehrere Ausgangssignale.
\class{SignalProcessor} stellt eine einheitliche Schnittstelle f"ur das Hinzuf"ugen und Entfernen von Ein- und Ausgangssignalen zur Verf"ugung und speichert die notwendigen Objektreferenzen dieser Signale.
Da sie aber nicht die konkrete Funktionalit"at der Signalverarbeitungsfunktion implementieren kann, ist \class{SignalProcessor} als abstrakte Klasse implementiert.

\class{SignalProcessor} "ubernimmt drei wesentliche Aufgaben implementierter Signalverarbeitungsfunktionen:
\begin{enumerate}
	{% Klammern damit "Anderungen lokal bleiben
	\renewcommand{\theenumi}{\arabic{enumi}}
	\renewcommand{\labelenumi}{{\theenumi}.}
	\item Bereitstellen einer einheitlichen Schnittstelle mit der alle Signalverarbeitungsfunktionen genutzt werden k"onnen
	\item Methoden zum Hinzuf"ugen, Entfernen und Speichern der Objektreferenzen der Ein- und Ausgangssignale
	\item einheitliche Dokumentation des Signalursprungs f"ur die Ausgabesignale
	}
\end{enumerate}
In den Methoden zum Hinzuf"ugen von Eingangs- und Ausgangssignalen wird sichergestellt, dass ein und dasselbe Signal mehrfach als Ein- bzw. Ausgangssignal verwendet wird.
Diese Funktionalit"at kann optional durch die implementierende Unterklasse verhindert werden.
Wird die Funktion \code{run()} aufgerufen, wird die durch die Unterklasse implementierte Signalverarbeitung durchgef"uhrt.
Zus"atzlich dazu wird durch \class{SignalProcessor} die Signalquelle der Ausgangssignale vermerkt.
Es werden bei der Ausf"uhrung einer Signalverarbeitungsfunktion die Programmversion, der Name der verarbeitenden Funktion und auch die genutzten Parameter in den Ausgangssignalen als Quelle vermerkt.
Damit wird erreicht, dass die Entstehungsgeschichte von verarbeiteten Signalen automatisiert aufgezeichnet wird.

In \picref{package_signalprocessing} sind drei abstrakten Methoden ersichtlich, die durch eine konkretisierende Unterklasse implementiert werden m"ussen.
Die Methoden \code{getName()} und \code{getParameterString()} dienen der Dokumentation der Verarbeitungsschritte und geben einen \code{String} mit Funktionsnamen (und eventuell notwendiger Versionsnummer) bzw. die genutzten Parameter der Funktion zur"uck.
Die dritte Methode \code{performFunctionality()} soll die konkrete Verarbeitung durchf"uhren und die Ausgangssignale entsprechende ver"andern.
Durch diese Umsetzung kann sich ein zuk"unftiger Entwickler auf die Implementierung des Algorithmus konzentrieren.
Alle organisatorischen Nebenabl"aufe sind durch \class{SignalProcessor} abgedeckt.

%- einmalig: einfach im Menu der zugeh"origen Konfigurationsdialog registrieren und er "ubernimmt die Arbeit

\subsection{Kontinuierliche Verarbeitung von Signalen durch \class{LiveSignalProcessor}}
\label{sec:live_signal_processor}

Aus \picref{package_signalprocessing} ist ersichtlich, dass die Klasse \class{LiveSignalProcessor} von \class{SignalProcessor} abgeleitet ist.
Somit wird auch die Funktionalit"at der einmaligen Signalverarbeitung mit "ubernommen.
Zus"atzlich dazu wird der Funktionsumfang dahingehend erweitert, dass eine kontinuierliche Signalverarbeitung ausgef"uhrt werden kann.
Um dieses dynamische Verhalten zu erm"oglichen wird durch die Klasse \class{LiveSignalProcessor} das \emph{Interface} \class{DataChangeListener} implementiert.
Der \class{LiveSignalProcessor} meldet sich bei allen Eingangssignalen als \emph{Observer} an und erh"alt dadurch Informationen "uber ver"anderte (Eingangs-) Daten.
Die gewonnenen Informationen werden, sofern der \class{LiveSignalProcessor} gestartet ist, an die implementierende Unterklasse weitergegeben, die diese dann verarbeiten kann.
Informationen ver"anderter Daten werden verworfen, wenn die kontinuierliche Signalverarbeitung gestoppt ist.
Der Zustand ob die Verarbeitung ein- oder ausgeschaltet ist, wird in der \code{boolean}-Variable \code{isRunning} gespeichert.
Der Zustand kann "uber die Methoden \code{start()} und \code{stop()} ein- bzw. ausgeschaltet werden.

Eine implementierende Unterklasse muss neben den im \secref{signal_processor} genannten Methoden \code{getName()}, \code{getParameterString()} und \code{performFunctionality()} noch zwei weitere Methoden implementieren: \code{preLiveAction()} und \code{handleEvent()}.
\code{handleEvent()} erh"alt als Parameter die auftretenden \class{DataChangeEvent}s und soll diese verarbeiten.
Die Methode \code{preLiveAction()} wird durch die Klasse \class{LiveSignalProcessor} aufgerufen, wenn die kontinuierliche Signalverarbeitung gestartet wird.
Mit diesem Zwischenschritt soll erm"oglicht werden, dass eventuell notwendige Vorbereitungen ausgef"uhrt werden k"onnen bevor das erste \code{DataChangeEvent} weitergegeben wird.
In dem implementiertem Beispiel \class{RRLiveCalculator} wird dieser Methodenaufruf genutzt, das Ausgabesignal f"ur das entsprechende Eingangssignal zu berechnen.

\begin{figure}[tbh]
\centering
\includegraphics[angle=-90, width=0.8\textwidth]{bilder/detail_LiveSignalProcessor_Menus_interaction.pdf}
\caption{Registrierung eines \class{LiveSignalProcessor}s im Men"u}
\label{pic:interaction_menu_lsp}
\end{figure}

Damit gestartete kontinuierliche Signalverarbeitungsfunktionen durch den Anwender wieder gestoppt werden k"onnen, ist eine automatisierte Men"u"anderung implementiert (siehe \picref{interaction_menu_lsp}).
Durch eine Dialogbox, die "uber das Men"u vom Anwender aufgerufen wird, werden die notwendigen Parameter der Signalverarbeitungsfunktion eingestellt.
Die Dialogbox erzeugt das \class{LiveSignalProcessor}-Objekt und "ubergibt die gew"ahtlen Parameter.
Nachdem der \class{LiveSignalProcessor} gestartet wurde, wird er im Men"u mit der Methode \code{registerStartedLiveSignalProcessor()} registriert.
Diese Registrierung hat zur folge:
\begin{itemize}
	\item Ein Men"ueintrag mit dem Namen des gestarteten \class{LiveSignalProcessor} wird erzeugt und in das Men"u eingef"ugt.
	\item Die laufende Signalverarbeitungsfunktion und der entsprechende Men"ueintrag werden zusammen in einem \class{RunningSignalProcessorMenuItem}-Objekt gespeichert.
\end{itemize}
Wird der entsprechende Men"ueintrag durch den Nutzer aufgerufen, wird die \code{stop()}-Methode des \class{LiveSignalProcessor}s aufgerufen und der Men"ueintrag wieder entfernt.

%\begin{figure}[tbh]
%\centering
%\includegraphics[angle=-90, width=0.8\textwidth]{bilder/verbesserung_LiveSignalProcessor_Menus_interaction.pdf}
%\caption{M"ogliche Verbesserung der Menu-Interaktion}
%\label{pic:interaction_menu_lsp_improved}
%\end{figure}

%- Klasse \class{LiveSignalProcessor}
%- Umsetzung wie in \picref{interaction_menu_lsp}
%- bestehende Funktionalit"at erkl"aren
%- Verbessern nach \picref{interaction_menu_lsp_improved}
%- vorgesehene Funktion erkl"aren
%- Vorteil: weniger Eingriff in die bestehende Struktur durch Implementierer notwendig

\subsection{Implementierung weiterer Signalverarbeitungsmethoden}

Das Paket \code{signalprocessing} ist daf"ur ausgelegt, um um zus"atzliche Signalverarbeitungsfunktionen einfach erweitert werden zu k"onnen.
Die Implementierungen der zwei Hauptklassen \class{SignalProcessor} und \class{LiveSignalProcessor} als abstrakte Klassen erm"oglicht ein schnelles einbinden neuer Funktionalit"aten.
Vier Schritte sind durch einen zuk"unftigen Entwickler auszuf"uhren:
\begin{itemize}
	\item Auswahl einer Superklasse abh"angig davon ob eine einfache oder kontinuierliche Signalverarbeitung stattfinden soll.
	\item Implementierung einer Unterklasse die die gew"unschte Funktionalit"at bereit stellt.
	\item Programmierung einer Dialogbox, die dem Benutzer des Programmes Einstellungen der Parameter erlaubt.
	\item Integration des Konfigurationsdialogs in das Men"u des Programms.
\end{itemize}
Es ist empfohlen alle Klassen in einem eigenem Paket einzubinden um Namenskollisionen zu unterbinden.
Das neue Paket kann und sollte als im bestehenden Paket \code{gst.signalprocessing} angelegt werden, damit die Gliederung der Pakete innerhalb des Programms nach Funktionalit"at erhalten bleibt.
Der Konfigurationsdialog soll nach der Erstellung und Initialisierung des konkreten Signalverarbeitungsobjektes die Verarbeitungsroutine starten (durch den Methodenaufruf von \code{run()}).
Wie schon im \secref{live_signal_processor} erw"ahnt, muss die kontinuierliche Verarbeitung durch die Methode \code{start()} ausgel"ost werden.
Das Abbrechen der kontinuierlichen Signalverarbeitung durch den Nutzer ist in der vorgelegten Implementierung bereits integriert.

%- Ableiten der neuen Implementierung von \class{SignalProcessor} bzw. \class{LiveSignalProcessor}
%- Implementierung eines Konfigurationsdialoges
%- Registrieren des Dialogs im Menu/ Erstellen eines Menueintrages

% EOF
