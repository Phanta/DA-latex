\section{[WIP]Signalverarbeitung}



\begin{figure}[htb]
\centering
\includegraphics[angle=-90, width=\textwidth]{bilder/package_signalprocessing.pdf}
\caption{"Ubersicht "uber das \class{signalprocessing}-Paket}
\label{pic:package_signalprocessing}
\end{figure}

- zwei Arten von Signalverarbeitern unterst"utzt: einmaliger und kontinuierlicher Ablauf

\subsection{[WIP]Einfache Signalverarbeitung durch \class{SignalProcessor}}

- einmalig: einfach im Menu der zugeh"origen Konfigurationsdialog registrieren und er "uber nimmt die Arbeit

\subsection{[WIP]Kontinuirliche Verarbeitung von Signalen}

\begin{figure}[tbh]
\centering
\includegraphics[angle=-90, width=0.8\textwidth]{bilder/detail_LiveSignalProcessor_Menus_interaction.pdf}
\caption{Registrierung eines \class{LiveSignalProcessor}s im Menu}
\label{pic:interaction_menu_lsp}
\end{figure}
\begin{figure}[tbh]
\centering
\includegraphics[angle=-90, width=0.8\textwidth]{bilder/verbesserung_LiveSignalProcessor_Menus_interaction.pdf}
\caption{M"ogliche Verbesserung der Menu-Interaktion}
\label{pic:interaction_menu_lsp_improved}
\end{figure}

- Klasse \class{LiveSignalProcessor}
- Umsetzung wie in \picref{interaction_menu_lsp}
- bestehende Funktionalit"at erkl"aren
- Verbessern nach \picref{interaction_menu_lsp_improved}
- vorgesehene Funktion erkl"aren
- Vorteil: weniger Eingriff in die bestehende Struktur durch Implementierer notwendig

\subsection{Implementierung weiterer Signalverarbeitungsmethoden}

- Ableiten der neuen Implementierung von \class{SignalProcessor} bzw. \class{LiveSignalProcessor}
- Implementierung eines Konfigurationsdialoges
- Registrieren des Dialogs im Menu/ Erstellen eines Menueintrages

% EOF
