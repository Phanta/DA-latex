\section{[WIP]Signalverarbeitung mit dem \texttt{signalprocessing}-Paket}

\subsection{"Uberblick "uber das \code{signalprocessin}-Paket}

Die Signalverarbeitung der Software wird durch das Paket \class{signalprocessing} erm"oglicht.
Der Schwerpunkt bei der Implementierung dieses Paket liegt in der Umsetzung einer erweiterbaren Plattform f"ur Signalverarbeitungsfunktionen.
Die Bereitstellung und Implementierung spezieller Signalverbeitungsfunktion stellen eine untergeordnete Rolle dar und soll an dieser Stelle nur beispielhaft erfolgen.
Als Musterbeispiel f"ur die Funktionalit"at der Signalverarbeitung wird die Berechnung von RR-Zeitreihen herangezogen.
Genauer soll aus den Zeitpunkten von Annotationen eines Kanals eine Zeitreihe der zeitlichen Abst"ande der gegebenen Annotationen automatisiert erstellt werden.
Es wird bewusst eine Signalverarbeitungsfunktion geringer Komplexit"at implementiert um den Fokus auf das allgemeine Zusammenspiel der beteiligten Klassen und nicht auf den Algorithmus selbst zu legen.

Zur Berechnung der Schlag-zu-Schlag-Intervalle werden folgende Schritte ausgef"uhrt:
\begin{enumerate}
	\item Auswahl eines Annotationskanals durch den Benutzer
	\item Berechnung der Zeitpunkte der Annotationen aus ihren Samplenummern und der (virtuellen) Abtastfrequenz
	\item Berechnung der Differenz der Zeitpunkte aufeinander folgender Annotationen
	\item Speichern der errechneten Zeitdifferenzen in einem durch den Benutzer benannten Signal
\end{enumerate}
Um die Darstellung der Zeitreihe zu erm"oglichen, wird jeder errechneten Differenz ein Zeitpunkt zugewiesen.
Das errechneten Datenpunkte werden als \class{Value} gespeichert.
Der Wert $V_{value}$ ist die Differenz der Zeitpunkte $t_{n+1} - t_n$ zweier Annotationen $A_n$ und $A_{n+1}$.
Der Zeitpunkt des \class{Value}s $t_{value}$ wird auf den Zeitpunkt $t_{n+1}$ der zweiten Annotation $A_{n+1}$ festgelegt.

\begin{figure}[htb]
\centering
\includegraphics[angle=-90, width=\textwidth]{bilder/package_signalprocessing.pdf}
\caption{"Ubersicht "uber das \code{signalprocessing}-Paket}
\label{pic:package_signalprocessing}
\end{figure}

In dem \code{signalprocessing}-Paket wird zwischen zwei Arten der Ausf"uhrung von Signalverarbeitungsfunktionen unterschieden (siehe \picref{package_signalprocessing}).
Die erste Art ist das einmalige Anwenden einer Verarbeitungsfunktion auf ein oder mehrerer Signale und ist im \secref{signal_processor} er"ortert.
Ein Anwendungsbeispiel daf"ur sind Filterfunktionen oder die Berechnung von RR-Intervallen aus bereits annotierten \ac{EKG}-Daten.
Die zweite Ausf"uhrungsart ist eine kontinuirliche Verarbeitung von Signalen.
Hierbei wird auf eine Ver"anderung des zu verarbeitenden Signals reagiert und die Ausgabe entsprechend angepasst.
Das implementierte Beispiel daf"ur ist die Berechnung und Anzeige der RR-Zeitreihe, w�hrend der Benutzer des Programms Annotationen ver"andert bzw. korrigiert.
Diese zweite Verarbeitungsart ist im \secref{live_signal_processor} weiter unten beschrieben.

%- zwei Arten von Signalverarbeitern unterst"utzt: einmaliger und kontinuierlicher Ablauf

\subsection{Einmalige Signalverarbeitung durch \class{SignalProcessor}}
\label{sec:signal_processor}

Die Klasse \class{SignalProcessor} stellt einen Prototypen einer allgemeinen Signalverarbeitungsfunktion dar.
Objekte dieser Klasse besitzen ein oder mehrere Eingangssignale und ebenso ein oder mehrere Ausgangssignale.
\class{SignalProcessor} stellt eine einheitliche Schnittstelle f"ur das Hinzuf"ugen und Entfernen von Ein- und Ausgangssignalen zur Verf"ugung und speichert die notwendigen Objektreferenzen dieser Signale.
Da sie aber nicht die konkrete Funktionalit"at der Signalverarbeitungsfunktion implementieren kann, ist \class{SignalPorcessor} als abstrakte Klasse implementiert.

\class{SignalProcessor} "ubernimmt drei wesentliche Aufgaben implementierter Signalverarbeitungsfunktionen:
\begin{enumerate}
	\item Bereitstellen einer einheitlichen Schnittstelle mit der alle Signalverarbeitungsfunktionen genutzt werden k"onnen
	\item Methoden zum Hinzuf"ugen, Entfernen und Speicherung der Objektreferenzen der Ein- und Ausgangssignale
	\item einheitliche Dokumentation des Signalursprungs f"ur die Ausgabesignale
\end{enumerate}
In den Methoden zum Hinzuf"ugen von Eingangs- und Ausgangssignalen wird sichergestellt, dass ein und dasselbe Signal mehrfach als Ein- bzw. Ausgangssignal verwendet wird.
Diese Funktionalit"at kann optional durch die implementierende Unterklasse auch ausgestellt werden.
Wird die Funktion \code{run()} aufgerufen, wird die durch die Unterklasse implementierte Signalverarbeitung durchgef"uhrt.
Zus"atzlich dazu wird durch \class{SignalProcessor} der Signalursprung der Ausgangssignale vermerkt.
Es werden bei der Ausf"uhrung einer Signalverarbeitungsfunktion die Programmversion, die Version der verarbeitenden Funktion und auch die genutzten Parameter in den Ausgangssignalen als Quelle vermerkt.
Damit wird erreicht, dass automatisiert die Entstehungsgeschichte von verarbeiteten Signalen aufgezeichnet wird.

In \picref{package_signalprocessing} sind drei Methoden als abstrakt ersichtlich, die durch eine konkretisierende Unterklasse implementiert werden m"ussen. Die Methoden \code{getName()} und \code{getParameterString()} dienen der Dokumentation der Verarbeitungsschritte und geben einen \code{Sring} mit Funktionsnamen (und eventuell notwendiger Versionsnummer) bzw. die genutzten Parameter der Funktion zur"uck.
Die dritte Methode \code{performFunctinality()} soll die konkrete Verarbeitung durchf"uhren und die Ausgangssignale entsprechende ver"andern.
Durch diese Umsetzung kann sich ein zuk"unftiger Entwickler auf die Implementierung des Algorithmus konzentrieren und alle organisatorischen Nebenabl"aufe sind schon durch \class{SignalProcessor} abgedeckt.

%- einmalig: einfach im Menu der zugeh"origen Konfigurationsdialog registrieren und er "ubernimmt die Arbeit

\subsection{[WIP]Kontinuirliche Verarbeitung von Signalen durch \class{LiveSignalPorcessor}}
\label{sec:live_signal_processor}

\begin{figure}[tbh]
\centering
\includegraphics[angle=-90, width=0.8\textwidth]{bilder/detail_LiveSignalProcessor_Menus_interaction.pdf}
\caption{Registrierung eines \class{LiveSignalProcessor}s im Menu}
\label{pic:interaction_menu_lsp}
\end{figure}
\begin{figure}[tbh]
\centering
\includegraphics[angle=-90, width=0.8\textwidth]{bilder/verbesserung_LiveSignalProcessor_Menus_interaction.pdf}
\caption{M"ogliche Verbesserung der Menu-Interaktion}
\label{pic:interaction_menu_lsp_improved}
\end{figure}

- Klasse \class{LiveSignalProcessor}
- Umsetzung wie in \picref{interaction_menu_lsp}
- bestehende Funktionalit"at erkl"aren
- Verbessern nach \picref{interaction_menu_lsp_improved}
- vorgesehene Funktion erkl"aren
- Vorteil: weniger Eingriff in die bestehende Struktur durch Implementierer notwendig

\subsection{Implementierung weiterer Signalverarbeitungsmethoden}

- Ableiten der neuen Implementierung von \class{SignalProcessor} bzw. \class{LiveSignalProcessor}
- Implementierung eines Konfigurationsdialoges
- Registrieren des Dialogs im Menu/ Erstellen eines Menueintrages

% EOF
