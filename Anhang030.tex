\chapter{Programm kompilieren mit \emph{Eclipse}}

Nachfolgende Schritte sollten ausgef"uhrt werden, damit das erstellte Programm mit der Entwicklungsumgebung \emph{Eclipse} kompiliert werden kann.
\begin{enumerate}
	{% Klammern damit "Anderungen lokal bleiben
	\renewcommand{\theenumi}{\arabic{enumi}}
	\renewcommand{\labelenumi}{{\theenumi}.}
	\item Importieren der Verzeichnisstruktur in den \emph{Workspace}:
		  "Uber den Men"upunkt \emph{File\.\chemarrow Import...} den Import-Dialog aufrufen.
		  Anschlie{\ss}end den Unterpunkt \emph{General\.\chemarrow File System} ausw"ahlen und mit \emph{Next} fortfahren.
		  Dann das auf dem Datentr"ager enthaltene Eclipse-Verzeichnis ausw"ahlen und mit \emph{Finish} best"atigen.
	\item Hinzuf"ugen der notwendigen Bibliotheken:
		  Aufrufen des Dialogs zum "Andern der Projekteinstellungen "uber den Men"upunkt \emph{Project\.\chemarrow Preferences}.
		  Dabei muss sichergestellt werden, dass das GST-Projekt im \emph{Package Explorer} ausgew"ahlt ist.
		  Unter \emph{Java Build Path\.\chemarrow Libraries} folgende Einstellungen vornehmen/"uberpr"ufen.
		\begin{enumerate}
			{% Klammern damit "Anderungen lokal bleiben
			\renewcommand{\theenumii}{\arabic{enumii}}
			\renewcommand{\labelenumii}{{\theenumii}.}
			\item Java Runtime Environment System Library: \emph{Add Library...\.\chemarrow JRE System Library\.\chemarrow Next\.\chemarrow Workspace default JRE\.\chemarrow Finish}
			\item Unisens Bibliotheken: \emph{Add JARs...\.\chemarrow GST/lib/unisens-2.0.1297/\textbf{org.unisens.jar}} sowie \emph{GST/lib/unisens-2.0.1297/\textbf{org.unisens.ri.jar}}
			\item JCommon und JFreeChart: \emph{Add JARs...\.\chemarrow GST/lib/JFreeChart-1.0.14/\textbf{jcommon-1.0.17.jar}} und \emph{GST/lib/JFreeChart-1.0.14/\textbf{jfreechart-1.0.14.jar}}
			}
		\end{enumerate}
	}
\end{enumerate}