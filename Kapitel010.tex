\chapter{Einleitung}

\section{Motivation}

Ergebnisse automatisierter Biosignalverarbeitungsmethoden werden aus mehreren Gr"unden oftmals manuell nachbearbeitet.
So erfordert die Entwicklung neuer Methoden h"aufig eine Verifikation der Ergebnisse und eine eventuelle Korrektur der automatisch generierten Ausgabe.
Zus"atzlich ist eine schnelle visuelle "Uberpr"ufung von Ergebnissen, um einen ersten Eindruck "uber den Effekt einer "Anderung an einer Methode zu bekommen, ein Mittel, das in der Entwicklungphase genutzt wird.
Daher besteht eine Notwendigkeit eines Werkzeugs, welches die Visualisierung "ubernimmt und den Entwickler beim Editieren von Messdaten und Ergebnissen der Signalverarbeitung unterst"utzt.

Ein solches Werkzeug kann durch die Definition und Festlegung von Ein- und Ausgabeformaten zu einer Vereinheitlichung von Datenformaten f"uhren.
Durch die Bereitstellung eines solchen Werkzeugs f"ur Dritte kann auch die methodische Grundlage f"ur die Kooperation verschiedener Institutionen geschaffen werden.
Um solche Kooperationen zu unterst"utzen sollte es, aufgrund der unterschiedlichen Voraussetzungen, wenig spezialsierte Anforderungen an seine Umgebung stellen.



\section{Zielstellung}
\label{sec:zielstellung}

Das Ziel dieser Arbeit ist ein Programm zu konzipieren und umzusetzen, das unterschiedliche (Bio-) Signale grafisch darstellt und dem Nutzer die M"oglichkeit bietet, Zeitpunkte und -intervalle innerhalb des Signalverlaufs zu markieren und mit Kommentaren zu versehen.
Hierbei soll insbesonderer die gleichzeitige Darstellung mehrerer Signale unterschiedlicher Natur und Auspr"agung unterst"utzt werden.
Die Erstellung und Bearbeitung von Markierungen soll leicht verst"andlich aus der \ac{GUI} heraus geschehen.
Zudem soll eine Grundlage geschaffen werden, paralell aufgenommene Signale in einem Datensatz zu vereinen.

Zus"atzlich soll eine zuk"unftige Erweiterung der Funktionalit"at erm"oglicht und unterst"utzt werden.
Daher ist eine klare Gliederung der Einzelkomponenten gefordert und die Dokumentation des Quelltextes sowie der einzelnen Programmteile fundamentaler Bestandteil der Aufgabenstellung.
Um die Erweiterbarkeit zus"atzlich zu verbessern, soll die sp"atere Einbindung von Methoden der Signalverarbeitung vorbereitet werden.
Daf"ur soll eine einfache Signalverarbeitungsfunktion in das Programm implementiert werden und in die Benutzeroberfl"ache integriert werden.
Die Arbeit eines zuk"unftigen Entwicklers wird somit durch die beispielhafte Integration einer zus"atzlichen Methode vereinfacht.

Neben der Entwicklerdokumentation soll auch eine seperate Dokumentation f"ur die Benutzer des Programms zur Verf"ugung gestellt werden.
In dieser Nutzerdokumentation soll dem Anwender die Funktionsweise und Bedienung des Programms verst"andlich gemacht werden.



\section{Konkretisieren der Aufgabenstellung}

Die Entwicklung automatiserter Datenverarbeitungsmethoden am \ac{IBMT} erfolgt haupts"achlich innerhalb der von \ml (MathWorks, Inc., Natick, MA, USA) bereit gestellten Umgebung statt.
Das zu erstellende Programm soll aber unabh"angig aus lizensrechtlichen Gr"unden von \ml unabh"angig sein.
Durch die M"oglichkeit in \ml Java-Code auszuf"uhren, bietet es sich an das Programm mit der Programmiersprache Java zu implementiern.
Dadurch werden zwei Ziele erreicht:
\begin{itemize}
	\item Ausf"uhrbarkeit des Programms auf Computersystem durch Nutzung frei erh"altlicher Software
	\item Integration der \ac{IBMT}-internen Nutzung des Programms in die bereits vorhandenen Entwicklungmethoden und -Werkzeuge
\end{itemize}
Die Implementierung in Java hat zus"atzlich noch den Vorteil, dass die Umsetzung der \ac{GUI} durch vorhandenen Bibliotheksfunktionen unterst"utzt und vereinfacht wird.
%Da jedoch die Datenspeicherung von \ml in einem propriet"aren Dateiformat erfolgt und dieses sich mit den \ml -Versionen in der Vergangenheit oft ver"andert hat, muss f"ur die Speicherung und den Transport von Daten ein geeignetes Format gew"ahlt werden.
Die Datenspeicherung durch \ml erfolgt in einem propriet"arem Dateiformat und unterlag in der Vergangenheit h"aufigen Ver"anderungen.
Daher soll f"ur die Speicherung und den Transport von Daten ein geeignetes, von \ml unabh"angiges Datensatzformat gew"ahlt werden.
Zus"atzlich m"ussen auch Skripte f"ur eventuell anfallenden Import- und Exportfunktionen der Daten bereit gestellt werden.
Bei der Erstellung der Konvertierungsskripte soll auf leicht verst"andliche und einfach Schnittstelle der Fokus gelegt werden.
Eine ausf"uhrliche Dokumentation der Schnittstelle und der Skripte selbst ist nat"urlich auch ein Kernpunkt.
%-- Programmiersprache Java
%-- notwendige Konvertierungsskripte (Matlab <-> Java) f"ur Daten

Wie schon in \secref{zielstellung} erw"ahnt, ist die Erweiterbarkeit des Programmes ein wesentlicher Punkt der Umsetzung.
Somit ist die Kapselung der einzelnen Programmkomponenten ein Bestandteil der Aufgabenstellung.
Daher ist es notwendig die einzelnen funktionellen Implementierungen voneinander unabh"angig zu gestalten.
Interaktionen zwischen den einzelnen Komponenten soll "uber einfach Schnittstellen erfolgen.
Zus"atzlich ist es notwendig Test zu implementieren, die die korekte Funktionsweise der einzelnen Komponenten verifizieren.
Damit kann nach einer ge"anderten Implementierung die Funktionalit"at erneut "uberpr"uft werden.
%-- Aufgabenteilung: Funktion
%-- zuk"unftige Entwicklung

Die \ac{GUI} soll eine intuitive Arbeit mit dem Programm erm"oglichen.
Da aber Begriffe wie ,,einfache Bedienbarkeit`` und ,,leicht verst"andliche Oberfl"ache`` von jeder Person anders interpretiert werden und somit nicht objektiv gemessen werden k"onnen, muss das Programm und seine Funktionalit"at validiert werden.
Hierzu sollen typische Arbeitabl"aufe mithilfe von Beispieldatens"atzen ausgef"uhrt werden.
Durch die Validierung kann aber nur festgestellt werden, ob die Software den W"unschen und Erwartungen des Nutzers gerecht wird.
Die im obigen Absatz erw"ahnte Verifikation der Funktionalit"at wird durch die Validierung der Software nicht obsolet.
%-- Benuzterschnittstelle
%-- Validierung mit Beispieldaten
%-- ungleich Verifikation (die geschieht mit Tests)

Um dem Nutzer die Arbeit mit dem Programm zu erm"oglichen, soll eine Benutzerdokumentation erstellt werden.
In dieser Dokumentation soll dem Nutzer die M"oglichkeiten, die ihm das Programm bietet, erl"autert werden.
Die einzelnen Werkzeuge und die Funktionsweise des Programmes sollen vermittelt werden.
Eine weitere Dokumentation soll f"ur die zweite Zielgruppe (zuk"unftige Entwickler des Programmes) erstellt werden.
Im Zusammenhang mit der vorliegenden Arbeit soll innerhalb des zweiten Dokumentes das Wissen vermittelt werden.
Genauer werden die einzelnen Bestandteile des Programmes und internen Beziehungen untereinander diskutiert.
Die Entwicklerdokumentation besteht aus drei Elementen:
\begin{itemize}
	\item Einem \verb|pdf|-Dokument, dass den allgemeinen Aufbau und das interne Zusammenspiel der Komponent erkl"art.
	\item Die mit dem Dokumentationswerkzeug Javadoc automatisiert erstellte Dokumentation aller Klassen, deren Membervariablen und Methoden, die die jeweilgen Schnittstellen im Detail beschreibt.
	\item Die Kommentierung des Quelltextes an notwendigen Stellen, die die Funktionsweise und Implementierung der Algorithmen besser verst"andlich machen.
\end{itemize}
Besonders die beispielhaft implementierte Signalverarbeitungsfunktion und die dazugeh"orige Dokumentation soll eine Weiterentwicklung des Programmes erm"oglichen.
Wird dieses Ziel erreicht, kann das Programm eine Grundlage f"ur eine Plattform automatiserter Signalverarbeitung bilden.
%-- Dokumentation f"ur Benutzer und Weiterentwickler
%-- m"ogliche Grundlage f"ur Programm der automatisierten Signalverarbeitung

%% EOF %%%%%%%%%%%%%%%%%%%%%%%%%%%%%%%%%%%%%%%%%%%%%%%%%%%%%%
