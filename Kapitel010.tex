\chapter{Einleitung}

\section{Motivation}

\section{Zielstellung}

\section{Unisens}

Das vom Forschungszentrum Informatik und Institut f�r Technik der Informationsverarbeitung der Universit�t Karlsruhe entwickelte Datenformat Unisens dient der Speicherung und der Dokumentation von Sensordaten.
Es ist konzipiert, Daten verschiedener Sensoren innerhalb eines Datensatzes zu speichern.
Ein Datensatz ist im Dateisystem durch ein eigenes Verzeichnis und eine Headerdatei \verb|unisens.xml| hinterlegt.
In der Headerdatei werden alle Informationen �ber die Bestandteile des Datensatzes, deren Formatierung und ihre semantischen Zusammenh�nge gespeichert.
Messwerte eines Sensors werden �blicherweise in einer Datendatei innerhalb des Verzeichnisses abgespeichert.
Eine solche Datendatei wird als \emph{entry} in dem Datensatz bezeichnet.
Als m�gliche Sensordaten werden sowohl kontinuierlich abgetastete Signale als auch ereignisorientierte Daten unterst�tzt.
Alle Metainformationen zu den Sensordaten werden in der Headerdatei abgspeichert, so dass die Datendateien immer nur die reinen Messdaten enthalten.

\subsection{Unterst�tzte Arten von Daten}

Unisens kennt vier Arten von Daten:
\begin{description}
	\item[Signale \emph{(signal)}] \hfill \\
		Signale sind kontinuierlich abgetastete, numerische Messdaten.
		Sie zeichnen sich durch eine beliebige aber konstante Abtastrate und ihre Abtastaufl�sung aus.
		Zudem k�nnen Signale aus mehreren Kan�len bestehen, die aber alle in ein und derselben Datei abgespeichert werden.
	\item[Ereignisse \emph{(event)}] \hfill \\
		Ereignisse sind diskrete Zeitpunkte die mit einem textlichen Beschreibung versehen sind. (z.B. Triggersignale)
		Sie zeichnen sich durch einen Zeitstempel und einer kurzen Beschreibung aus.
		Optional k�nnen noch Kommentare zu einem Ereignis hinzugef�gt werden.
	\item[Einzelwerte \emph{(value)}] \hfill \\
		Einzelwerte sind eine Kombination der beiden oben genannten Datenarten.
		Sie beinhalten numerische Werte die zu bestimmten Zeitpunkten aufgenommen wurden.
		Mit ihnen ist es m�glich Daten zu speichern, die nicht in festen Zeitintervallen gemessen werden.
	\item[Propriet�re Daten \emph{(custom data)}] \hfill \\
		Mit dieser Art k�nnen anwendungsspezifisch Daten gespeichert werden, die durch die drei oben genannten Arten nicht erfasst werden k�nnen.
\end{description}

\subsection{Implementierungsdetails}

In diesem Abschnitt wird kurz auf einige Details der Umsetzung des Unisens-Formates eingegangen.

Die Zeitpunkte von Ereignisdaten und Einzelwertdaten werden �ber eine virtuelle Abtastrate bestimmt.
Der Zeitpunkt eins jeden \emph{Event}- oder \emph{Value}-Eintrags ist als ganzzahlige Samplenummer dieser Abtastrate gespeichert.
Die Zeit eines Ereignisses, relativ zum Messbeginn, errechnet sich somit aus dem Produkt $Samplenummer / Abtastrate$.
M�chte man die M�glichkeit Ereignisse f�r jeden beliebigen Datenpunkt eines Datensatzes zuordnen zu k�nnen, dann muss die virtuelle Abtastrate als das kleinste gemeinsame Vielfache aller vorhandenen Abtastraten gew�hlt werden.

...

%% EOF %%%%%%%%%%%%%%%%%%%%%%%%%%%%%%%%%%%%%%%%%%%%%%%%%%%%%%
