\chapter{Einleitung}

\section{Motivation}

Ergebnisse automatisierter Biosignalverarbeitungsmethoden werden aus mehreren Gr"unden oftmals manuell nachbearbeitet.
So erfordert die Entwicklung neuer Methoden h"aufig eine Verifikation der Ergebnisse und eine eventuelle Korrektur der automatisch generierten Ausgabe.
Zus"atzlich ist eine schnelle visuelle "Uberpr"ufung von Ergebnissen, um einen ersten Eindruck "uber den Effekt einer "Anderung an einer Methode zu bekommen, ein Mittel, das in der Entwicklungphase genutzt wird.
Daher besteht eine Notwendigkeit eines Werkzeugs, welches die Visualisierung "ubernimmt und den Entwickler beim Editieren von Messdaten und Ergebnissen der Signalverarbeitung unterst"utzt.

Ein solches Werkzeug kann durch die Definition und Festlegung von Ein- und Ausgabeformaten zu einer Vereinheitlichung von Datenformaten f"uhren.
Durch die Bereitstellung eines solchen Werkzeugs f"ur Dritte kann auch die methodische Grundlage f"ur die Kooperation verschiedener Institutionen geschaffen werden.
Um solche Kooperationen zu unterst"utzen sollte es, aufgrund der unterschiedlichen Voraussetzungen, wenig spezialsierte Anforderungen an seine Umgebung stellen.

\section{Zielstellung}

Das Ziel dieser Arbeit ist ein Programm zu konzipieren und umzusetzen, das unterschiedliche (Bio-) Signale grafisch darstellt und dem Nutzer die M"oglichkeit bietet, Zeitpunkte und -intervalle innerhalb des Signalverlaufs zu markieren und mit Kommentaren zu versehen.
Hierbei soll insbesonderer die gleichzeitige Darstellung mehrerer Signale unterschiedlicher Natur und Auspr"agung unterst"utzt werden.
Die Erstellung und Bearbeitung von Markierungen soll leicht verst"andlich aus der \ac{GUI} heraus geschehen.
Zudem soll eine Grundlage geschaffen werden, paralell aufgenommene Signale in einem Datensatz zu vereinen.

Zus"atzlich soll eine zuk"unftige Erweiterung der Funktionalit"at erm"oglicht und unterst"utzt werden.
Daher ist eine klare Gliederung der Einzelkomponenten gefordert und die Dokumentation des Quelltextes sowie der einzelnen Programmteile fundamentaler Bestandteil der Aufgabenstellung.
Um die Erweiterbarkeit zus"atzlich zu verbessern, soll die sp"atere Einbindung von Methoden der Signalverarbeitung vorbereitet werden.
Daf"ur soll eine einfache Signalverarbeitungsfunktion in das Programm implementiert werden und in die Benutzeroberfl"ache integriert werden.
Die Arbeit eines zuk"unftigen Entwicklers wird somit durch die beispielhafte Integration einer zus"atzlichen Methode vereinfacht.

Neben der Entwicklerdokumentation soll auch eine seperate Dokumentation f"ur die Benutzer des Programms zur Verf"ugung gestellt werden.
In dieser Nutzerdokumentation soll dem Anwender die Funktionsweise und Bedienung des Programms verst"andlich gemacht werden.

\section{Konkretisieren der Aufgabenstellung}

-- Programmiersprache Java
-- notwendige Konvertierungsskripte (Matlab <-> Java) f"ur Daten

-- Aufgabenteilung: Funktion
-- zuk"unftige Entwicklung
-- Benuzterschnittstelle

-- Validierung mit Beispieldaten
-- ungleich Verifikation (die geschieht mit Tests)

-- Dokumentation f"ur Benutzer und Weiterentwickler
-- ein Schwerpunkt Erweiterbarkeit
-- m"ogliche Grundlage f"ur Programm der automatisierten Signalverarbeitung

%% EOF %%%%%%%%%%%%%%%%%%%%%%%%%%%%%%%%%%%%%%%%%%%%%%%%%%%%%%
